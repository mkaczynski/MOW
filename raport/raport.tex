\documentclass[a4paper,12pt]{article}

\usepackage[utf8]{inputenc}
\usepackage{polski}
\usepackage{fullpage}
\usepackage{hyperref}

\title{Proste algorytmy klasyfikacji tekstu \\ Raport końcowy}
\author{Marcin Kaczyński \and Krystian Lieber}

\begin{document}

\maketitle

\section{Opis zadania} 

W ramach zadania - ''Proste algorytmy klasyfikacji tekstu (TF-IDF, naiwny klasyfikator Bayesowski, kNN). Porównania ze standardowymi algorytmami klasyfikacji dostępnymi w R.'' zaimplementowano 3 wymienione klasyfikatory które następnie zostały wykorzystane w zagadnieniu rozpoznawania reklam i niechcianych wiadomości SMS.\\

Do nauki i oceny klasyfikatorów wykorzystano zbiór \textit{SMS Spam Collection}\footnote{http://archive.ics.uci.edu/ml/datasets/SMS+Spam+Collection} z repozytorium UCI. Opis zbioru znajduje się w sekcji \ref{dane:charakterystyka}.

\section{Algorytmy}

W sekcji znajduje się opis wykorzystywanych algorytmów. W stosunku do informacji zawartych w dokumentacji wstępnej, opis został rozszerzony o 
informacje i decyzje związane z implementacją. Zgodnie z wymaganiami, algorytmy zostały zaimplementowane w języku \textbf{R}.

\subsection{Algorytm \textit{k-NN}}

opisz

\subsection{Algorytm \textit{TF-IDF}}

mk opisze

\subsection{Naiwny klasyfikator Bayesa}

mk opisze

\subsection{Algortym z R}

myślę, że bez szczegółów, tylko stwierdzenie co, z jakiego pakietu + referencja do opisu

\subsection{Algorytm II z R}

\section{Opis eksperymentów}

\subsection{Pytania, na które poszukiwano odpowiedzi}

Celem zadania było odpowiedzenie na pytanie, czy przekazana wiadomość SMS jest pożądaną, zwykłą wiadomością, czy zawiera niepożądane treści lub reklamy.

\subsection{Charakterystyka zbioru danych}\label{dane:charakterystyka}

Wybrany zbiór danych (\textit{SMS Spam Collection}) jest zbiorem etykietowanych wiadomości które zostały zebrane na potrzeby badania spamu wysyłanego SMS'ami. Korpus składa się z $4827$ zwykłych wiadomości oraz z $747$ wiadomości zawierających spam. Cała kolekcja wiadomości jest umieszczona w jednym pliku, w którym każda linijka składa się z etykiety wiadomości którą poprzedza oraz z wiadomości w oryginalnej postaci. Poniżej kilka przykładów:

\begin{itemize}
	\item ham What you doing?how are you? 
	\item ham dun say so early hor... U c already then say... 
	\item ham MY NO. IN LUTON 0125698789 RING ME IF UR AROUND! H* 
	\item spam FreeMsg: Txt: CALL to No: 86888 \& claim your reward of 3 hours talk time to use from your phone now! ubscribe6GBP/ mnth inc 3hrs 16 stop?txtStop 
	\item spam Sunshine Quiz! Win a super Sony DVD recorder if you canname the capital of Australia? Text MQUIZ to 82277. B 
	\item spam URGENT! Your Mobile No 07808726822 was awarded a L2,000 Bonus Caller Prize on 02/09/03! This is our 2nd attempt to contact YOU! Call 0871-872-9758 BOX95QU 
\end{itemize}

\subsection{Preprocessing zbioru danych}

W ramach wstępnego przetwarzania danych, zostały wykonana następujące kroki:

\begin{itemize}
\item z tekstów zostały wycięte znaki interpunkcyjne i znaki specjalne,
% \# \& \' \( \) \* \+ \, \- \. \/ \: \; \< \= \> \? \@ \[ \\ \] \^ \_ \` \{ \| \} \~ \% \$ \- \! \"
\item z tekstów zostały wycięte wszystkie liczby,
\item wielokrotnie występujące białe znaki zostały zastąpione pojedynczymi spacjami,
\item zostały usunięte słowa z listy stopwords dla języka angielskiego zdefiniowanej w pakiecie \textit{tm} R.
\end{itemize}

Implementacja przewiduje również usunięcie znaczników xml z tekstów, ale w przypadku badanego zbioru nie występuje potrzeba usuwania takich znaczników. W stosunku do wcześniejszch rozważań, nie wykonano rozwijania skrótów pojawiających się w wiadomościach, ani nie włączono ich do listy stopwords.

\section{Parametry algorytmów} % parametry algorytmów, których wpływ na wyniki był badany,

\subsection{Algorytm \textit{k-NN}}

opisz

\subsection{Algorytm \textit{TF-IDF}}

mk opisze

\subsection{Naiwny klasyfikator Bayesa}

mk opisze

\section{Ocena jakości modeli} % sposób oceny jakości modeli,

Modele dla opisanych algorytmów były budowane z wykorzystaniem $70\%$ danych ze zbioru \textit{SMSSpamCollection}. Dane do budowy modeli zostały
wybrane w sposób losowy, z zachowaniem proporcji w klasach (tzn. wybierano dokładnie $70\%$ danych przypisanych do klasy \textit{spam} i $70\%$ 
danych z klasy \textit{ham}. Pozostałe $30\%$ wykorzystano do przetestowania jakości budowanych modeli. Algorytmy były testowane z wykorzystaniem
tych samych zbiorów danych uczących i testowych, w celu zachowania możliwie dużego stopnia obiektywności oceny.\\

Modele były oceniane w taki sposób, że dla każdej próbki ze zbioru testowego, porównywano rzeczywistą klasę do której przypisana jest próbka, z
wartością zwracaną przez algorytm. W ramach oceny uwzględniono oddzielnie jakość oceny klasyfikacji próbek ze zbioru \textit{spam} (iloraz niechcianych widomości do ich łącznej liczny) oraz \textit{ham} (analogicznie), oraz ogólny wskaźnik jakości (iloraz prawidłowo sklasyfikowanych próbek do wszystkich próbek w zbiorze testowym).

\section{Wyniki}

podejślij w jakiejś surowej postaci, to wkleję żeby było jednolicie

\section{Wnioski}

jak masz wnioski to wklejaj, a jak nie to coś tu naściemniam

\end{document}
